\documentclass[12pt]{article}
\usepackage{latexsym}
\usepackage{amssymb,amsmath}
\usepackage[pdftex]{graphicx}
\usepackage{color}
\usepackage{hyperref}


\topmargin = 0.1in \textwidth=5.7in \textheight=8.6in

\oddsidemargin = 0.2in \evensidemargin = 0.2in


\begin{document}\textsl{}

\begin{center}
COMPUTER SCIENCE 20, SPRING 2014 \\

Module \#18 (Digraphs and Relations)
\end{center}
Author: Roger Huang\\
Reviewers: Anupa Murali\\
Last modified: March 8, 2014
\paragraph*{Readings from Meyer}
\begin{itemize}
\item Meyer section 9.1, 9.3, 9.4, 9.10, 9.11
\end{itemize}

\paragraph*{Executive Summary}
\begin{enumerate}
\item Properties of binary relations
\begin{itemize}
\item \textit{Transitive}: A binary relation $R$ on the set $A$ is transitive iff\\$u R v \wedge v R w \implies u R w$ for all $u,v,w \in A$.
\item \textit{Reflexive}: $u R u$ for all $u \in A$.
\item \textit{Irreflexive}: $\neg(uRu)$ for all $u \in A$
\item \textit{Symmetric}: $u R w \implies w R u$ for all $u,w \in A$.
\item \textit{Antisymmetric}: $u R w \implies \neg(wRu)$ for all $u, w \in A$, $u \neq w$.
\item \textit{Asymmetric}: $u R w \implies \neg(wRu)$ for all $u, w \in A$.
\end{itemize}
\item Recall that $G$ is a binary relation on $V$, where $uGw$ means that there is an edge from $u$ to $w$.
\begin{itemize}
\item $G^+$ is transitive and is the \textit{transitive closure} of $G$. This means that $G^+$ is the minimal transitive relation that includes $G$ (i.e. $G \subseteq G^+$).
\item $G^*$ is reflexive, transitive, and the \textit{reflexive transitive closure} of $G$.
\end{itemize}
\item The vertices $u,v \in V$ are \textit{strongly connected} iff $uG^*v \wedge vG^*u$. That is, if there exists a walk from $u$ to $v$ and a walk back from $v$ to $u$.
\item Special types of relations
\begin{itemize}
\item \textit{Strict partial orders}: transitive and irreflexive
\item \textit{Weak partial orders}: transitive, reflexive, and antisymmetric
\item \textit{Equivalence relations}: transitive, reflexive, and symmetric
\item A relation $R$ is a weak partial order iff $R = D^*$ for some DAG $D$
\item A relation $R$ is a equivalence relation iff $R$ is the strongly connected relation of some digraph
\end{itemize}
\item An equivalence relation $R$ decomposes the domain into subsets called \textit{equivalence classes}, where $aRb$ iff $a$ and $b$ are in the same equivalence class.
\end{enumerate}

\paragraph*{Check-in problem}
\begin{enumerate}
\item Let $R$ be a binary relation on Harvard students such that $aRb$ iff $a$ and $b$ are people in the same house. Then $R$ is (check all that apply):
\begin{enumerate}
\item Reflexive
\item Transitive
\item Symmetric
\item Antisymmetric
\end{enumerate}
\end{enumerate}

\paragraph*{Small group problems}

\begin{enumerate}
\item Draw a directed graph with 3 vertices $A, B, C$ representing a relationship that is:
\begin{enumerate}
\item Reflexive
\item Symmetric
\item Antisymmetric
\item Transitive
\end{enumerate}
\item Prove that if a relation $R$ is transitive and irreflexive, then it is asymmetric.
\item Say that a string $x$ overlaps a string $y$ if there exist strings $p,q,r$ such that $x = pq$ and $y = qr$, with $q \neq \epsilon$. For example, $abcde$ overlaps $cdefg$, but does not overlap $bcd$ or $cdab$. Answer each of the following questions and prove your answer.
\begin{enumerate}
\item Is the overlap relation reflexive?
\item Is it symmetric?
\item Is it transitive?
\end{enumerate}
\item Determine what properties each of the following relations have. For those that are equivalence relations, briefly describe what the equivalence classes are in the relation.
\begin{enumerate}
\item The relation ``shares a class with'', where two people share a class if there is a class they are both enrolled in this semester.
\item The relation $R$ on $\mathbb{Z}$, where $aRb$ if $b$ is a multiple of $a$.
\item The relation $R$ on $\mathbb{Z} \times \mathbb{Z}$, with $(a,b)$ $R$ $(c,d)$ if $ad = bc$. 
\end{enumerate}
\iffalse
\item Recall that any weak partial order is $D^*$ for some DAG $D$. Let the weak partial order divisibility ($a|b$ means $b$ is divisible by $a$) be $D^*$.
\begin{enumerate}
\item Draw the corresponding DAG for vertices $\{1, 2, ... 12\}$, where an edge from $a$ to $b$ means that $a|b$. Use as few edges as possible.
\item Add the vertex $24$ to this graph. What is the smallest number of additional edges you need to do this?
\end{enumerate}
\fi
\end{enumerate}
\pagebreak


\end{document}

