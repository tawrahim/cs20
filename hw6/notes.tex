\documentclass[12pt]{article}
\usepackage{latexsym}
\usepackage{amssymb,amsmath}
\usepackage[pdftex]{graphicx}
\usepackage{color}
\usepackage{hyperref}


\topmargin = 0.1in \textwidth=5.7in \textheight=8.6in

\oddsidemargin = 0.2in \evensidemargin = 0.2in


\begin{document}\textsl{}

\begin{center}
COMPUTER SCIENCE 20, SPRING 2014 \\
Module \#15 (Recursive Data Types and Structural Induction) - in class\\
\end{center}
\begin{itemize}
\item Meyer, introductory section in Chapter 6, and section 6.1.\\
6.2-6.4 (optional) provide concrete examples of recursive data types and structural induction (from real definitions in programming languages).
\end{itemize}

\paragraph*{Executive Summary}
\begin{enumerate}

\item Recursive definitions. Remember how to recursively define a set $S$.
\begin{itemize}
\item Base case(s). These define the base cases of $S$.
\item Constructor case(s). These define new elements of $S$ from previously constructed elements of $S$ or from the base cases of $S$.
\item Examples:
	\begin{itemize}
	\item Strings: Let $A$ be a nonempty set called an \textit{alphabet}, whose elements are \textit{characters}. Let the set of strings $A^*$ be defined as follows:
		\begin{itemize}
		\item Base case: the empty string $\epsilon$ is in $A^*$.
		\item Constructor case: If $a \in A$ and $s \in A^*$, then $(a,s) \in A^*$.
		\item If $A = \{a,b,\ldots,y,z\}$, the string $(x,(y,(z,\epsilon)))$ would be a member of $A^*$.
		\end{itemize}
	\item String Concatenation: Let the concatenation $s \cdot t$ of strings $s,t \in A^*$ be defined as follows:
		\begin{itemize}
		\item Base case: $\epsilon \cdot t ::= t$.
		\item Constructor case: $(a,s) \cdot t ::= (a,s \cdot t)$.
		\end{itemize}
	\end{itemize}
\end{itemize}

\item Structural induction. This proof technique is used for proving properties of recursively defined data types.
\begin{itemize}
\item Let $S$ be a inductively defined set. Let $P(x)$ be a property we're trying to prove about $S$ for all $x \in S$.
\item Base case(s): For each base case $x$ in the definition of $S$, prove $P(x)$.
\item Constructor case(s): For each construction case that uses $x_1, \ldots, x_n \in S$ to construct $x \in S$, show that
$$P(x_1), \ldots, P(x_n) \implies P(x)$$
In structural induction proofs, the inductive hypothesis is that the property $P$ holds true for all ``sub-cases'' of $x$: $P(x_1),\ldots,P(x_n)$.
\end{itemize}

\end{enumerate}
\end{document}
