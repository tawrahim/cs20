\documentclass[12pt]{article}
\usepackage{latexsym}
\usepackage{amssymb,amsmath}
\usepackage[pdftex]{graphicx}
\usepackage{color}
\newcommand{\ra}{\rightarrow}
\topmargin = 0.1in \textwidth=5.7in \textheight=8.6in
\oddsidemargin = 0.2in \evensidemargin = 0.2in

\begin{document}
\begin{center}
COMPUTER SCIENCE 20, SPRING 2014 \\
\smallskip

Module \#29 (Convergent and Divergent Series)
\end{center}
Author: Yifan Wu\\
Reviewer: Nick Longenbaugh\\
Last modified: April 24, 2014 

\paragraph*{Executive Summary}
\begin{enumerate}

\item Common series
\begin{enumerate}
\item The geometric series: $\sum_{i=0}^{\infty} q^i = \frac{1}{1-q}$, provided $|q| < 1$.
\item The negative binomial series: $\sum_{i=0}^{\infty} \binom{i+r-1}{r-1} q^i = \frac{1}{(1-q)^r}$ if $|q| < 1$.
\item The ``harmonic series'' $\sum_{i=1}^{\infty} \frac{1}{i}$ is  divergent, but $\sum_{i=1}^n \frac{1}{i^2} = \frac{\pi^2}{6}$.
\item The ``exponential series''  $\sum_{i=0}^{\infty} \frac{\alpha^i}{i!} = e^{\alpha}$. 
\end{enumerate}

\end{enumerate}

\paragraph*{Small group problems}


\begin{enumerate}

\item Paul offers to let you play a game.  He'll flip a fair coin until he flips tails, then pay you $2^k$ dollars, where $k$ is the number of heads he flipped.  For instance, the sequence $HHHT$ would earn you \$8, while a tails on the first flip would leave you with just \$1.  What is the expected value of your winnings if you play this game once?  How much would you be willing to pay Paul to play this game one time?

\item Simplify $\frac{1}{1 \cdot 2} + \frac{1}{2 \cdot 3} + \frac{1}{3 \cdot 4} + \ldots + \frac{1}{99 \cdot 100}$.

\item Prove that the``harmonic series'' $1 + \frac{1}{2} + \frac{1}{3} + \cdots$ is divergent.

\emph{Hint:}  try to show that $\forall N > 0, \exists m\text { such that} \sum_{i=N+1}^m \frac{1}{i} > \frac{1}{2}$ (why does this prove divergence?)

\item Simplify $\left(1 + \frac{1}{a}\right)\left(1 + \frac{1}{a^2}\right)\left(1 + \frac{1}{a^4}\right) \cdots \left(1 + \frac{1}{a^{2^{100}}}\right)$.

\emph{Hint:}  What could you multiply the given product by on the left that would help you simplify this expression?  The formula for the ``difference of two squares'' may help you here.

\item \textbf{Challenge:}  Let $S$ be the set $\{1, 2, 3, \ldots, 9, 11, 12, \ldots, 19, 21, \ldots\}$ consisting of all natural numbers that do not contain the digit zero.
Do you think that the sum of the reciprocals of the elements of $S$ converges or diverges?  Justify your answer.



\end{enumerate}

\end{document}
