\documentclass[12pt]{article}
\usepackage{latexsym}
\usepackage{linguex}
\usepackage{amssymb,amsmath}
\usepackage[pdftex]{graphicx}

\topmargin = 0.1in \textwidth=5.7in \textheight=8.6in

\oddsidemargin = 0.2in \evensidemargin = 0.2in

\newif\ifsolutions
%\solutionstrue
\solutionsfalse

\begin{document}

\begin{center}
\large
COMPUTER SCIENCE 20, SPRING 2014 \\
\medskip

Module \#12 (Relations and Functions)
\end{center}
Author: Nick Longenbaugh\\
Reviewer: Ruth Fong\\
Last modified: February 22, 2014 (added ``total'' to question 3)
\medskip


\paragraph*{Executive Summary}
\begin{enumerate}
\item A {\em binary relation} describes relationships from one set $A$, called the {\em domain}, to another set $B$, called the {\em codomain} through a subset of $A \times B$ called the {\em relation graph}.  This is often depicted as a diagram with arrows from elements of the {\em domain} to elements of the {\em codomain}.
\begin{itemize}
\item A {\em function} is a special case of a relation in which each member of the domain has at most one arrow coming out of it.  Most of the time when we see functions, they can be described in a concise way, such as $f(x) = x^2$.
\item A relation is {\em surjective} when every item in the codomain has at least one arrow coming in -- that is, every element in the codomain is covered.
\item A relation is {\em total} when every item in the domain has at least one arrow coming out of it -- that is, every element in the domain participates in the relation.
\item A relation is {\em injective} when every element of the codomain has at most one arrow coming in -- that is, if you start with an element in the codomain that has an arrow, there's no ambiguity where in the domain it came from.
\item A relation is {\em bijective} when every element of the domain has exactly one arrow pointing out, and every element of the codomain has exactly one arrow pointing in.
\end{itemize}
\end{enumerate}


\paragraph*{Check in question}
\begin{enumerate}
\item Indicate which of the following pairs of sets has a surjective function from the first set in the pair to the second.
\begin{enumerate}
\item $(\{1,2,3\}, \{a,b\})$
\item $(\{1,a,b\}, \{2,3,4,5\})$
\item $(\{1\}, \{ \})$
\end{enumerate}
\end{enumerate}
\end{document}
