\documentclass[12pt]{article}
\usepackage{latexsym}
\usepackage{amssymb,amsmath}
\usepackage[pdftex]{graphicx}


\topmargin = 0.1in \textwidth=5.7in \textheight=8.6in

\oddsidemargin = 0.2in \evensidemargin = 0.2in


\begin{document}

\begin{center}
    COMPUTER SCIENCE E-20, SPRING 2014 \\
    Homework Problems\\
    Pigeonhole, Proofs, Induction I \\
    Author: Tawheed Abdul-Raheem
\end{center}

\smallskip

\begin{enumerate}

    \item Gauss's tomb is in the shape of a 17-sided star. You learn that he left proofs of theorems in seven different vertices. Prove that there must exist a sequence of five consecutive vertices that contains at least three theorems.

        Cryptic hint: each theorem is five pigeons, and $35 > 34$.

        \textbf{Solution: } 
        From our problem set, we know that there are 34 possibilities to choose a five consecutive vertices this is derived
        from choosing in this order {\em 17 * 2 * 1 * 1} which can singinfy our pigeon holes. We also know that we have
        35 pigeons, this is derived from our seven vertices and our five possibilities. If each theorem is five pigeons and we know from our combinations that there are 4 vertices ({\em Pigeonholes}). We can conclude that there must exist a sequence of five consecutive vertices that contains at least three theorems.

    \item The gamekeeper of a wild animal park has a square of area 4.5 square miles in which he would like to place his ten leopards. These cats are highly territorial, and if two of them are within a mile or less of one another, they will fight. Show that there is no way to place the leopards without causing a fight to occur.

        \textbf{Solution: } 
        We have 10 leopards and each leopard needs to be away from the other leopard by at least a mile. We know that the area of our square is
        {\em 4.5} miles and each side of our our square is $\sqrt{4.5}$ $\approx$ {\em 2.12132}. 
        Inorder for there not to be any fight we need at an area
        of {\em 10} miles to accomodate the spacing. Because our area is small there is the gurantee that there will be fight since {\em 8} of the
        leopards would be placed in an approximation of less than a mile.

    \item Prove by contradiction that if $ab = n$ where $a$, $b$, and $n$ are nonnegative integers, then $a$ or $b$ (or both) must be less than or equal to $\sqrt{n}$.

        \textbf{Solution: } To proof by contradiction we suppose that the claim is false 
        \\ we can say that  $ab$ $\neq$ $n$, since our original prosition claims that $a$, $b$, and $n$ are nonnegative integers, then $a$ or $b$ (or both) must be less than or equal to $\sqrt{n}$. We can derive that the following: $a$ \textgreater $\sqrt{n}$ and $b$ \textgreater $\sqrt{n}$. To solve we can take the square of the equations and have ${a}^{2}$ \textgreater $n$ and also ${b}^{2}$ \textgreater $n$ and so we come up with ${a}^{2}$ ${b}^{2}$ = ${n}^{2}$, we take the square root of botsides and this leaves us with the original claim that $ab$ = $n$

    \item Prove by contradiction that $\sqrt{3}$ is irrational. Then explain why your proof is no longer valid if you replace 3 by an arbitrary positive integer $n$.


        \textbf{Solution: } Suppose that $\sqrt{3}$ is rational and can be written as a fraction in the form of $n/d$ in the lowest terms. Squaring both sides gives ${n}^{2}$/${d}^{2}$ = 3 and so $3{d}^{2}$ = ${n}^{2}$ This implies that $n$ is a multiple of 3, therefore ${n}^{2}$ must be a multiple of 9. So $d^2 = n^2/3$ and $d^2$ is divisible by 3. That means that $d$ and $n$ are both divisible by 3 so they were not in their lowest terms. If we replace $2$ with say $4$ the contradiction is not going to hold

    \item \begin{enumerate}

            \item  Prove that for all nonnegative integers $n$
                \[ \sum_{i=0}^{n}i^{3}=\left( \sum_{i=0}^{n}i \right)^{2} \]
                {\em Hint: the following identity may be useful}
                \[ \sum_{i=0}^ni=\frac{n(n+1)}{2} \]
                \textbf{Solution: }
                To prove the theorem, define predicate $P(n)$ to be the equation. Now the theorem can restated for as the claim that $P(n)$ is true for all all nonnegative $n$ 
                \[ \sum_{i=0}^{n}i^{3}=\left( \sum_{i=0}^{n}i \right)^{2} \]

                \textbf{base case: } We prove that $P(0)$ is true
                \[P(0) = \sum_{i=0}^{n}i^{3}=\left( \sum_{i=0}^{n}i \right)^{2} \]
                \[ \implies \sum_{i=0}^{0}i^{3}=\left( \sum_{i=0}^{0}i \right)^{2} \]
                \[ \implies 0 = 0 \]

                \textbf{Prove that: } $P(n)$ implies $P(n + 1)$ for every nonnegative integer $n$
                \[ P(n + 1) \implies \sum_{i=0}^{n+1}i^{3}=\left( \sum_{i=0}^{n+1}i \right)^{2} \]

                \[ \text {Let L1} = \sum_{i=0}^{n}i^{3} \]
                \[ \text {L1} = \sum_{i=0}^{n}i^{3} + (n + 1)^3 \]
                \[ \text {L1} = \left( \sum_{i=0}^{0}i \right)^{2} + (n + 1)^3\]
                \[ \text {L1} = \left(\frac{n(n+1)}{2} \right)^2 + (n + 1)^3 \]
                \[ \text {L1} = (\frac{n^2(n+1)^2}{4}) + (n + 1)^3 \]
                \[ \text {L1} = (n + 1)^2 (\frac{n^2}{4} + (n + 1))\]
                \[ \text {L1} = (n + 1)^2 (\frac{n^2 + (n+1)4}{4})\]
                \[ \text {L1} = (n + 1)^2 (\frac{n^2 + 4n + 4)4}{4})\]
                \[ \text {L1} = (n + 1)^2 (\frac{(n + 2)^2}{4})\]

                \[ \text {Let L2} = \sum_{i=0}^{n}i^{2} \]
                \[ \text {L2} = (\frac{(n + 1)(n + 2)}{2})^2\]
                \[ \text {L2} = \frac{(n + 1)^2(n + 2)^2}{4}\]
                \[ \therefore \text {L1} = \text {L2} \]
            \item An application: you receive an email from Klingon that invites you to help in diversifying the Klingon economy by making an investment in a new company that is commercializing the first Klingon-invented cryptographic algorithm. A document describing the algorithm is attached. It begins:

                ``Choose a large prime number $p$ that is the sum of the cubes of seven consecutive integers."

                What is a good mathematical reason for not investing?

                \textbf{Solution: } 
                \[ \sum_{i=0}^{n+7}i^{3}= \sum_{i=0}^{n+7}i - \sum_{i=0}^{n}i^{3}\]
                \[ = (\sum_{i=0}^{n+7}i)^{2} - (\sum_{i=0}^{n}i^{3})^{2} \]
                \[ = (\frac{(n+7)(n+8)}{2})^{2} - (\frac{n(n+1)}{2})^{2} \]
                \[ = (\frac{n^2+15n+56}{2})^{2} - (\frac{n^2+n}{2})^{2} \]
                \[ = (\frac{n^2+15n+56}{2} - \frac{n^2+n}{2}) * (\frac{n^2+15n+56}{2} + \frac{n^2+n}{2}) \]
                \[ = (\frac{14n+56}{2}) * (\frac{2{n}^2+16n+56}{2}) \]
                \[ = (7n+28)(n^2+8n+28) \]
                \[ = 7(n+4)(n^2+8n+28) \]
                From the above step we conclude that the solution is multiple of 7 so it is not a prime number

        \end{enumerate}

\end{enumerate}

\pagebreak

\end{document}
