\documentclass[12pt]{article}
\usepackage{latexsym}
\usepackage{linguex}
\usepackage{amssymb,amsmath}
\usepackage[pdftex]{graphicx}

\topmargin = 0.1in \textwidth=5.7in \textheight=8.6in

\oddsidemargin = 0.2in \evensidemargin = 0.2in

\newif\ifsolutions
%\solutionstrue
\solutionsfalse

\begin{document}

\begin{center}
\large
COMPUTER SCIENCE 20, SPRING 2014 \\
Module \#12 (Relations and Functions) - Checkin \\
Author: Tawheed Abdul-Raheem
\medskip

\end{center}
\medskip


\paragraph*{Small group questions}
\begin{enumerate}

\item Determine which labels apply to the following relations: function, total, injective, surjective, and bijective.

\begin{enumerate}
\item The relation that assigns a Harvard undergraduate student to a residential House \\
     surjective

\item The relation that maps every natural number to its square (from natural numbers to natural numbers) \\
    total, function
\item The relation that maps every integer to its square (from integers to natural numbers) \\
    total function 
\item The relation that maps every student in the class to every class he or she is taking this current semester. \\
    total, surjective
\end{enumerate}
\item Show that if two finite sets $A$ and $B$ are the same size, and $r$ is a total injective function from $A$ to $B$, then $r$ is also surjective; i.e. $r$ is a bijection. \\
    If two finite sets $A$ and $B$ have the same size or said to have the same cardinality $|A| = |B|$ and $r$ is a total injective frunction from $A$ to $B$, this means that $every$ element in the codomain has at most one arrow coming in. By proof, this statement also holds for a surjective relationship. This is because for a relationship to surjective everything in the codomain has something mapped to it, which happens to be satisfied by the fact that we know that the relationship is injective. --Hmm, I don't think my proof is true because an injective function can have elements in codomain with nothing pointing to it. \\
\item Given two functions $f,g$, let $g \circ f$ denote function composition, i.e., $g \circ f=g(f(x))$. Prove or disprove the following claim: If $f$ is total and injective and $g$ is total and surjective, then $g \circ f$ is injective. \\
    Suppose $f: A \rightarrow B$ is a total injective function and $f: B \rightarrow C$ is a total surjective function. To prove that $g \circ f: A \rightarrow C$ is injective, we need to prove that 
\item We have been representing English quantificational determiners like \emph{every} and \emph{some} as first order quantifiers. A more linguistically accurate representation of these words is to treat them as relations on sets:
\begin{itemize}
\item every $\equiv R$, where $R(A,B)=1$ iff $A \subseteq B$.
\item some $\equiv S$, where $S(A,B)=1$ iff $A\cap B \neq \emptyset$.
\end{itemize}
With these definitions in place, we can represent the semantics of a simple sentence as follows.
\ex. \a. Every boy likes Mary
\b. $R(B,L)=1$, where $B=\{x: x \text{ is a boy}\}$ and $L=\{y: y \text{ likes Mary}\}$.

Write the semantic formulae for the following English sentences. You may define predicates as necessary following the model established above.
\begin{enumerate}
\item A girl saw John. \\
    $S(G,J)=1$ where $G=\{x: x \text{ is some girl}\}$ and $J=\{y: y \text{ saw John}\}$.
\item Every MiG shot some pilot.\\
    $R(M,S(A,B)=1)=1$ where $M=\{x: x \text{ a mig}\}$ and $S(A,B)=\{y: y \text{ shot some pilot}\}$.
\end{enumerate} 
\item Prove that the composition of two surjective function is surjective\\
    Suppose two functions $f: A \rightarrow B$ and $g: B \rightarrow C$, to proof that $A \rightarrow C$, we need to show
    $\forall c \in C \exists a \in A $ such that $( g \circ f)(a) = c$ \\
    Since $g : B \rightarrow C$ is surjective, $\exists b \in  B$ such that $g(b) = c$ \\
    Since $f : A \rightarrow C$ is surjective, $\exists a \in  A$ such that $f(a) = b$ \\
    $g(f(a)) = g(b) = c$
\end{enumerate}
\end{document}
