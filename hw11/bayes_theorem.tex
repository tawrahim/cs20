\documentclass[12pt]{article}
\usepackage{latexsym}
\usepackage{amssymb,amsmath}
\usepackage[pdftex]{graphicx}
\usepackage{color}

\topmargin = 0.1in \textwidth=5.7in \textheight=8.6in
\oddsidemargin = 0.2in \evensidemargin = 0.2in

\begin{document}
\begin{center}
COMPUTER SCIENCE 20, SPRING 2014 \\
\smallskip

Module \#28 (Bayes Theorem \& the Monty Hall problem)
\end{center}
Author: Nick Longenbaugh\\
Reviewer: Steve Komarov\\
Last modified: April 14, 2014 

%\paragraph*{Reading from Meyer}
%You may have skimmed some of these sections earlier, but this time read them in more detail.
%\begin{itemize}
%\item Section 4.1 discusses sets.
%\item Section 4.5 discusses finite sets.
%\end{itemize}

\paragraph*{Executive Summary}
\begin{enumerate}

\item Conditional probability
\begin{itemize}
\item From last class: $P(A | B) = \frac{P(A \cap B)}{P(B)}, \text{ provided } P(B) > 0.$

\item Bayes' Rule (version 1): a tautology, given our definition:  $$P(A | B) = \frac{P(B | A)P(A)}{P(B)}$$

\item Bayes' Rule (version 2): $$P(A | B) = \frac{P(A)P(B | A)}{P(B | A)P(A) + P(B|\overline{A})P(\overline{A})}$$
\end{itemize}

\item Conditioning on a partition - useful when conditional probabilities are given
\begin{itemize}
\item If events $B_1, B_2, \cdots B_n$ are disjoint, and their union is the entire sample space $S$, then

$P(A) = P(A | B_1)P(B_1) +  P(A | B_2)P(B_2) +  \cdots + P(A | B_n)P(B_n).$
\item Useful special case: $P(A) = P(A | B)P(B) +  P(A | \overline{B})(1-P(B)).$
\item Finding the right partition is often the secret to solving a conditional probability problem.
\item Conditioning on a set of events whose union is not the entire sample space is a celebrated way to go wrong.
\end{itemize}
\item Generalized Monty Hall problems

\begin{itemize}
\item Behind some of the doors are prizes; behind the others are booby prizes. All arrangements of the prizes and booby prizes are assumed equally likely.
\item The contestant chooses a door.
\item Monty Hall, who knows where the prizes are located, opens a door, different from the one chosen by the contestant, behind which he knows there is a booby prize. (If there were multiple prizes, he could choose to reveal a prize).
\item Monty Hall then invites the contestant to switch his choice to one of the unopened doors. The question is whether making the switch increases the contestant's chance of winning a prize.
\item Conditioning on the event $C=$ ``there is a prize behind the contestant's original door'' usually makes these problems trivial.
\end{itemize}
\end{enumerate}
%\paragraph*{In-class exercise}
%\medskip
%\begin{enumerate}

%\item You are playing a variant of poker in which two cards from your opponent's hand (out of a total of five cards) are dealt face-up. The opponent has two aces showing. What is the (conditional) probability that he has four of a kind?
%\end{enumerate}

\paragraph*{Small group problems}
\begin{enumerate}
\item Wisdom of Solomon

In ancient Jerusalem, true prophets tell the truth 90\% of the time, while false prophets tell the truth half the time. Solomon's servant has brought him a group of three prophets, two true, one false. Each of the prophets knows which is which, but Solomon does not. Solomon asks prophet 1, ``Is prophet 2 a true prophet?''

\begin{enumerate} \item If the answer is ``No,'' what is the conditional probability that prophet 3 is a true prophet? 


\item If the answer is ``Yes,'' what is the conditional probability that prophet 2 is a true prophet? 

Hint for part a. 

As events, use $A_3$ = ``prophet 3 is the false prophet'' and $B$ = ``Solomon receives the answer `No.''' When calculating $P(B)$, partition the sample space into $A_3$ and the other two equally probable events\\ $A_2$ =``prophet 2 is the false prophet'' and\\ $A_1$ =``prophet 1 is the false prophet.''\\ Once you know $P(A_3|B)$, the problem is solved.

\end{enumerate}
\item An ambitious preschool, whose aim is to prepare toddlers for
Harvard, has 8 students, 4 boys and 4 girls.  Following the
example of Harvard, it divides its students randomly into two
groups of 4, which it names ``Lowell House'' and ``Eliot House.''
\begin{enumerate}
\item What is the probability $P_4$ that all four boys end up in
Eliot House?

\item What is the probability $P_3$ that three boys and one girl
end up in Eliot House, with the other boy and three girls in
Lowell House?

\item What is the probability $P_2$ that optimal diversity is
achieved, with two boys in Eliot House and two in Lowell House?

\item Verify that the sum of probabilities for all the ways of
dividing the class between the houses is correct.

\item A newly hired teacher meets one student, chosen at random,
from Eliot House.  The student is a boy.  Calculate the
conditional probabilities, given this event, that there are 4, 3,
2, 1, or 0 boys in Eliot House. (These conditional probabilities
should also sum to 1.)

\item Next the teacher meets a randomly chosen student from Lowell
House, who turns out to be a girl. Calculate the conditional
probabilities, given both this event and the previous one, that
there are 4, 3, 2, 1, or 0 boys in Eliot House.
\end{enumerate}
\item   In the admissions office at Monty Hall University there
are four interviewers. Three of them are
friendly, while the fourth is unfriendly. Every morning the
Dean of Admissions assigns them randomly to offices 1, 2, 3, and
4, with an equal probability for each possible assignment. A
student arrives for an interview and is asked to select which
office he wants to be interviewed in. He chooses office 1 and
learns that the interviewer in there is busy for the next half
hour.
``While you are waiting,'' says the Dean to the student, ``I would like you to meet one of our friendly interviewers.  He opens an office door (but not office 1) and introduces the student to a friendly interviewer. He then continues, ``Rather than waiting, would you prefer to be interviewed by someone in one of the other offices?"

If the student accapts this offer, does his probability of getting a friendly interviewer increase or decrease?

\item After the sinking of the Titanic, many
unidentified bodies were brought to Halifax for burial. The policy used was that anyone who was wearing a crucifix or
holding a rosary was buried in the Catholic cemetery, while
everyone else was buried in the non-Catholic cemetery. 

Assume the following: \begin{itemize} \item 20\% of unidentified
Titanic victims were Catholic. \item The conditional probability
that a Catholic on a sinking ship will have a crucifix or rosary
is 90\%.
 \item The conditional probability that a non-Catholic on a sinking ship will have a crucifix or rosary is 15\%.
 
\end{itemize}
\begin{enumerate}
\item Calculate the conditional probability that someone buried under a Titanic grave marker in the Catholic cemetery is in fact Catholic. 

\item Calculate the conditional probability that someone buried under a Titanic grave marker in the non-Catholic cemetery is a Catholic.   \end{enumerate} 

\end{enumerate}
\end{document}
