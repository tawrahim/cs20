\documentclass[12pt]{article}
\usepackage{latexsym}
\usepackage{amssymb,amsmath}
\usepackage[pdftex]{graphicx}
\usepackage{color}


\topmargin = 0.1in \textwidth=5.7in \textheight=8.6in

\oddsidemargin = 0.2in \evensidemargin = 0.2in


\begin{document}

\begin{center}
COMPUTER SCIENCE 20, SPRING 2014 \\

\smallskip

Module \#24 (General Principles of Counting)
\end{center}
Author: Steve Komarov\\
Reviewer: Paul Handorff\\
Last modified: \today

\medskip

\paragraph*{Executive Summary}
\begin{enumerate}

\item Fundamental rules for counting

\begin{itemize}

\item The sum rule: If sets $A$ and $B$ are disjoint, $A$ has $a$ elements, and $B$ has $b$ elements, their union has $a + b$ elements. This permits a ``divide and conquer'' strategy.

\item The difference rule: Suppose set $B$ has $b$ elements and its subset $A$ has $a$ elements. Then the set difference $B - A$ has $b-a$ elements. This rule is useful when it is easy to count sets $B$ and its subset $A$, but there is no simple way to count $B-A$ directly.

\item The product rule:
Suppose that set $A$ has $m$ elements and set $B$ has $n$ elements.  The Cartesian product $A \times B$ is the set of ordered pairs $(a, b)$, where $a$ is an element of $A$ and $b$ is
an element of $B$.  The set $A \times B$ has $mn$ elements.

\item The power rule: If $A$ has $n$ elements and you select a sequence of $r$ elements with replacement from $A$, there are $n^r$ possible sequences. 


\item The division rule:
When counting, it is fine to count things more than once, provided
that everything gets overcounted the same number of times and that
you know that number and divide by it.

\end{itemize}


\end{enumerate}

\pagebreak



\pagebreak

\paragraph*{In-class problems}

\begin{enumerate}

\item A license plate consists of either 
\begin{itemize}
\item 3 letters followed by 3 digits (standard plate)
\item 5 letters (vanity plate)
\item 2 characters - letters or numbers (big shot plate)
\end{itemize}
Compute the number of different possible license plates.

\item How many of the billion numbers in the range from 1 to $10^9$ (inclusive) contain the digit 1?

\item How many anagrams does the name ``hardy'' have? How about the names ``littlewood'' and ``ramanujan?'' The anagrams do not have to be dictionary words. 


\item \begin{enumerate}
\item A dodecahedron has 12 faces, each a regular pentagon. How many edges does it have?
\item Three faces of the dodecahedron meet at each vertex.  How many vertices does the dodecahedron have?



\end{enumerate}



\item {\bf Challenge:} How many binary strings of length 8 contain either four consecutive zeros or three consecutive ones. 




\end{enumerate}

\pagebreak

\end{document}
