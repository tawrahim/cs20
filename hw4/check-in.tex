\documentclass[12pt]{article}
\usepackage{latexsym}
\usepackage{amssymb,amsmath}
\usepackage[pdftex]{graphicx}
\usepackage{color}


\topmargin = 0.1in \textwidth=5.7in \textheight=8.6in

\oddsidemargin = 0.2in \evensidemargin = 0.2in

\begin{document}

\begin{center}
COMPUTER SCIENCE 20, SPRING 2014 \\
Module \#9 (Quantificational Logic I) Check-in \\
Author: Tawheed Abdul-Raheem
\end{center}

\medskip


Let the domain of discourse be members of the class and let $L(x,y)$ be the proposition ``$x$ likes $y$." Write the following colloquial English sentences using quantificational formulas. The sentences are not necessarily unambiguous. If a sentence has more than one possible meaning, explain the ambiguity and which interpretation you have chosen.

\begin{enumerate}

\item Everyone in the class likes some other member of the class. \\
    $\forall x \exists y, (x \not = y \wedge L(x,y))$

\item Someone doesn't like anybody and nobody likes that person. \\
    $\exists x \forall y, (x \not = y \wedge \neg L(x,y) \wedge \neg L(y,x))$

\item At least three different people like the same person. \\
    $\exists w \exists x \exists y \exists z, (w \neq z \wedge x \neq z \wedge y \neq z \wedge L(w,z) \wedge L(x,z) \wedge L(y,z))$

\end{enumerate}
\end{document}
